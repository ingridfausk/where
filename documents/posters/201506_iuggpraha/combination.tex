GEOSAT combines VLBI, SLR, GNSS, and DORIS data at the observation level as follows.

First, a 24h data set (an arc) (SLR: 1 week) is sorted in batches of
passes (FIGURE 1) for each geodetic technique. A pass is defined by observing a specific radio source or satellite from one or several specific ground or satellite based receivers. A transition from one pass to the next is defined
by a switch of one or more objects that are involved in the
computation of the residual. For instance, for VLBI the objects are
two stations and one radio source, for SLR the objects are one station
and one satellite. Each pass has its own copy of the state vector so
that it is possible to re-estimate all or parts of the state vector
for each pass as part of the next step: reducing the residuals by
least-squares estimation of some subset of 
parameters (clock, troposphere, range biases, ...). However, by increasing the state vector dimension it is
also possible to allow parameters to have an impact on the calculated
observations for time spans that are not restricted by the length of a
pass.

In the residual reduction step, in which the different observation techniques
are treated separately, outliers are removed and the impact of
discontinuities like VLBI clock breaks on the residuals are
reduced. This is essential for the next computation step in which the
techniques are combined sequentially, observation epoch by observation
epoch, by a UD (Upper Diagonal) Kalman filter. Especially, since sequential filters are more
susceptible to outliers than batch algorithms.

In the combination step it is important to note that the full state
vector and its a posteriori variance-covariance matrix are updated at
each epoch according to the postulated observation noise. In this way
information from one technique can resolve the parameter correlations
of a different technique in a natural way since these are described by
the full variance-covariance matrix. For instance, if we truncate the a posteriori variance-covariance matrix for each technique separately (for instance VLBI or SLR) before combination,
then important technique specific correlations, and therefore
information on the weaknesses of the individual solutions, could be lost. These
weaknessess of the solution are commonly re-established by applying a
Helmert type transformation to it before combination with other data
types.

As a first approach we will combine SLR and VLBI data so that the only parameters that
are shared by both techniques are Earth Rotation Parameters. However, in order to exploit that we
account for all correlations while combining the data types epoch by epoch it is possible that we
need a stronger connection between SLR and VLBI. Therefore we consider
implementing local ties at collocation sites as the first observation
in all data sets. An issue here will be, when added as
constraints in all data sets, and not only once as they should, the effective constraint imposed by the local
ties will be unrealistically strong.

When the sequential processing of data from all techniques within the
specified data set ($N_\mathrm{e}$ epochs) is finished we are left with an a posteriori state
vector, which is derived without any neglectance
of parameter correlations. The end state vector will typically consist
of $N_\mathrm{g}$ global parameters that are constant across the data set, like
station coordinates, and $N_\mathrm{s}$ stochastic parameters, describing the
troposphere or clock model errors, at the last epoch. Ideally, we
would then have backward filtered (smoothed) the state vector and
variance-covariance matrix to establish a larger matrix and state vector with dimension
\begin{equation}\nonumber
N=N_\mathrm{g}+N_\mathrm{s}\,N_\mathrm{e},
\end{equation}
formally containing all available information. However, using the complete variance-covariance information in the combination of data sets (arcs) (FIGURE 2), to obtain solutions spanning years,
is not feasible since the effective state vector will contain millions $\sim{}N_\mathrm{s}\,N_\mathrm{e}$ of parameters.

\endinput
