At the moment we work on challenges connected to the combination of VLBI and SLR epoch by epoch. One important issue in this respect will be how we tie the VLBI and SLR network together. For instance, it will be important that the networks are closely connected at an early stage (during epoch by epoch processing with UD Kalman filter) so that the techniques may fully complement each other, before we truncate the state vector and its variance-covariance matrix for the accumulated combination solution. To what extent will the Earth Rotation Parameters, shared by both techniques, connect the different networks and, alternatively, how will we use local ties? Shall we use all available local ties, or only the most reliable?

The output of each data set will be a state vector consisting of global (station coordinates) and stochastic parameters (clock, troposphere) at the last epoch of the data set. Although no approximations regarding correlations have been made at this stage, the solution's a posteriori variance-covariance matrix does not contain all available information on parameter correlations. If we use only this limited information (state and covariance at last epoch), as we intend to due to practical reasons (see Combination), it is possible that we during day by day combination have to involve Helmert transformations to describe the weaknesses of daily solutions.       

In the future we will include GNSS and DORIS in our processing chain, which we anticipate will require the introduction of new parameter types to achieve intra-technique consistency.    


\endinput
