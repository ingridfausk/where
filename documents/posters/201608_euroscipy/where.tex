Where is a software currently under development at the Norwegian Mapping
Authority. Where will be able to process and analyse data from the space
geodetic techniques VLBI, SLR, GNSS and DORIS, and combine these together to
provide very precise positioning of geodetic stations around the world.

{\bfseries The International Terrestrial Reference Frame}

The International Terrestrial Reference Frame (ITRF) is a coordinate system for
the Earth, which is necessary for climate monitoring where we want to track
minuscle movements over long time periods and for tracking satellites as they
orbit the Earth. Every few years the ITRF is updated through a collective effort
of several international research institutions. One of the goals of Where is to
be a tool that can support the further development of the ITRF.

{\bfseries Python}

As in many other research fields, Fortran is still heavily used in Geodesy. When
we started work on Where we opted for using Python as our main programming
language instead. Our motivations mainly came from Python's flexibility and ease
of use, combined with the rich ecosystem that exists for doing data analysis as
well as interfacing with established C and Fortran libraries.

Where is using the core of the Python datastack, including \texttt{numpy},
\texttt{scipy}, \texttt{pandas} and \texttt{matplotlib}, for a lot of
functionality. In addition, we rely on more specialized packages like
\texttt{astropy} and \texttt{jplephem} where special handling of geodetic data
is needed. Furthermore, we are custom building important parts of the
architecture, including a powerful caching mechanism and a flexible
datastructure for handling the data.

{\bfseries The datastructure}

We need an effective, flexible and powerful datastructure to handle the big
amounts of data related to one model analysis, and have chosen to implement
something we call a \texttt{where.dataset}. The \texttt{where.dataset} organizes
data in rows each representing one observation and fields with data for that
observation. Fields can be of different datatypes, where some are simply based
on existing types like strings, \texttt{numpy}-arrays or
\texttt{astropy}-times. Other fields are more specialized like a
\texttt{position}-datatype that is represented by \texttt{numpy}-3-vectors, but
knows how to transform themselves to other coordinate systems or calculate
things like azimuth and elevation to other \texttt{position}s. The
\texttt{where.dataset} is stored to disk using HDF5 (\texttt{h5py}) and JSON.
Accessing fields can be done both
using \texttt{dict}-notation: \texttt{dset['field']}
and \texttt{property}-notation: \texttt{dset.field}.

\endinput
