Let us look at a few examples on how to run the Where software. One of our
guiding principles is that the software should be as easy to use as possible.

{\bfseries A VLBI analysis}

To start a VLBI analysis, we simply need to provide the date we want to analyse:

\texttt{> where 2016 6 22 --vlbi}

Doing this will show the default configuration for VLBI analyses, including
options like which models to run, which ephemerides (positions for planets etc)
to use and which output to create. All of these options can be changed, or we
can accept the defaults and run the analysis.

As the model runs, all intermediary calculations are stored, allowing us to
later look at data at each step. For instance can we compare raw observation
numbers with calculated results, or look at the residuals before or after
estimation. The simplest way to do this is to use the There GUI-companion:

\texttt{> there}

\includegraphics[width=\linewidth]{figure/vlbi_20160622_nyales20}

{\itshape Residuals based on models (no estimation) from a VLBI analysis for the
  station at Ny-{\AA}lesund, Norway. Colors indicate the other station in the
  baseline.}

After a model run, we can also explore the data interactively. In practice, this
means that we get access to the datasets and can use the Python datastack to
manipulate them:

\texttt{> where 2016 6 22 --interactive}

\begin{verbatim}
Available datasets:
    v0, vlbi_calculate_XA_1
    v1, vlbi_calculate_XA_2
    v2, vlbi_edit_data_XA_0
    [...]

In [1]:  v1.residual
array([ 0.47276345, -0.05613388,
       -0.50012711, ...,  0.05031952,
       -0.00214976,  0.04906757])

In [2]: v1.unique('station')
['BADARY', 'MEDICINA', 'NOTO',
 'NYALES20', 'SVETLOE', 'WETTZELL',
 'YEBES40M', 'ZELENCHK']
\end{verbatim}

The final output of the analysis can be stored in many different formats. Sinex
is typically used when sharing analysis with others. However, one interesting
output format Where supports is Jupyter Notebooks. These allow us to for
instance run each model independently, and easily create live demonstrations of
the software highlighting particular parts of the analysis.

{\bfseries A GPS analysis}

Doing a GPS analysis is done in the same way as the VLBI analysis. We only need
to specify a different technique:

\texttt{> where 2016 6 22 --gnss}

The technique is specified as \texttt{--gnss} because Where will support
analyses of several of the systems, including GLONASS and Galileo, in addition
to GPS. Running Where like this will again bring up all the default options. For
GNSS these options will now include how to handle satellite orbits and which
stations to analyse.


\endinput
