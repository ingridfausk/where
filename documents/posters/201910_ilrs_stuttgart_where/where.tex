
{\large\bfseries Geodesy in Norway}\\

The Norwegian Mapping Authority has increased its contributions
on global reference frames over the last decade.  We are currently building a
new fundamental station at Ny-{\AA}lesund supporting VLBI, SLR, GNSS and DORIS.
We have been involved in the establishment of a UN resolution on global geodesy
and reference frames~\cite{un_ggrf}, and will work on the establishment of a
Global Geodetic Centre of Excellence, as proposed by the UN Subcommittee on
Geodesy. 

The Where software project completes the norwegian contribution, by
giving us a new tool for the analysis of geodetic data and giving us more
insight into such analysis along the way. 


{\large\bfseries How Where works}\\

Where is mainly implemented in Python. The Python ecosystem for data science is
very rich and Where utilizes several well known packages such as
\texttt{numpy}, \texttt{scipy} and \texttt{matplotlib}, as well as more
specialized packages like \texttt{jplephem}~\cite{jplephem}. In addition,
Python easily interfaces with other languages like C and Fortran, and Where
uses the \texttt{SOFA}~\cite{sofa_software} and
\texttt{IERS}~\cite{iers_software} Fortran libraries directly.

Using Python has several advantages, including rapid development, flexible and
easy to read code. Where is a command line tool, and comes with a configuration
file that can be used to customize the behavior of the program. This makes the
program quite easy to use, while at the same time having the possibility to use
different models and input for the analysis.

The results can be inspected using the graphical tool called There that has
been developed alongside Where (Figure~\ref{fig:there}).


{\large\bfseries The Where Workflow}\\

First, observation data are read and edited, and stored in an internal data
format. We use hdf5, which is a hierarchical data format compatible with the
dictionary structure in Python. Next, for SLR, the orbit and the theoretical
range observations are calculated, before station positions and EOP's are
estimated on the basis of those theoretical ranges.

The implementation of the individual models follow the 2010 IERS
Conventions~\cite{iers2010}, and when possible we have used software libraries
made available at the IERS web page~\cite{iers_software}. Table~\ref{tbl:models}
gives an overview of the models implemented in Where. The estimation of EOP
parameters and station positions is done using a Kalman
filter~\cite{mysen2017}. We use continuous piecewise linear functions for the
estimation.


{\large\bfseries Orbit determination}\\

We have implemented all models for orbit determination, see
Table~\ref{tbl:models}. Force models, variational equations and orbit
estimation is computed following~\cite{montenbruck2012}, and we have
implemented a Cowell orbit integrator following~\cite{oesterwinter1972}. The
observation equation for SLR is following~\cite{beutler2005}.  


{\large\bfseries Computation time}\\

The original Python program was slow, but we have done some improvements. Using
\texttt{numpy}, doing most operations on vectors (instead of sequentially) and
caching results when possible gives us a fairly fast program. Running Where for
one week of SLR analysis with the four satellites Lageos1, Lageos2, Etalon1 and
Etalon2 now takes about 18 minutes. We have programmed the orbit computation in
Cython, a programming language which gives us the combined power of Python and
C. It lets us tune readable Python code into plain C performance by adding
static type declarations in Python syntax. 

\endinput
