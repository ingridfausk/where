Where is a new software for geodetic analysis, currently being developed at the
Norwegian Mapping Authority (Kartverket). Where is based on our experiences with
the GEOSAT software~\cite{kierulf2010}, and will be able to analyse and combine
data from VLBI, SLR, GNSS and DORIS.


{\large\bfseries Why?}\\

The last decade the Norwegian Mapping Authority has increased its contributions
on global reference frames.  We are currently building a new fundamental station
at Ny-{\AA}lesund with VLBI, SLR, GNSS and DORIS. We have contributed to passing
a UN resolution on global geodesy and the importance of Global Geodetic
Reference Frames~\cite{un_ggrf}. The Where project is the third leg in this
effort, developing a new tool for the analysis of geodetic data and giving us
more insight into such analysis.


{\large\bfseries How?}\\

Where is mainly implemented in Python. The Python ecosystem for data science is
very rich and Where utilizes several well known packages such as \texttt{numpy}
and \texttt{scipy}, as well as more specialized packages
like \texttt{astropy}~\cite{astropy2013}
and \texttt{jplephem}~\cite{jplephem}. Data can easily be visualized with
and \texttt{matplotlib}. In addition, Python smoothly interfaces with other
languages like C and Fortran, which allows Where to use
the \texttt{SOFA}~\cite{sofa_software} and \texttt{IERS}~\cite{iers_software}
Fortran libraries directly. Python is also Open Source and freely available on
all major platforms.


{\large\bfseries What?}\\

Where works on the basis of a pipeline. First, observation data are read and
edited. Editing can for instance involve discarding bad observations or applying
an elevation cut off angle. Next, theoretical delays are calculated and a
quadratic clock polynomial is estimated. Station positions and other target
parameters such as EOP can then be estimated next. Finally, results are written
to disk in proper formats (Figure~\ref{fig:architecture}).

The implementation of the individual models follow the 2010 IERS
Conventions~\cite{iers2010}, and when possible we have used software libraries
made available at the IERS web page~\cite{iers_software}. Table~\ref{tbl:models}
gives an overview of the models available for Where. The estimation is done
using a Kalman filter with a Modified Bryson-Frazier smoother. Currently we are
using continuous piecewise linear functions to model the clock and troposphere
~\cite{mysen2017b}.


{\large\bfseries Where?}\\

A VLBI model consistent with current conventions is fully implemented.
Comparisons with other softwares (see box below) yield promising results. The
estimation module is implemented, but further testing is needed to get a good
solution. Where can read both NGS and vgosDb files, but requires that the
ionosphere and ambiguities are already solved.


{\large\bfseries When?}\\

We are an associated analysis center within the IVS and a report of our
activities the last two years will soon be published~\cite{kirkvik2017}. Our
short term plans are to finish a stable version of Where capable of producing
normal equations that can be used for IVS products~\cite{mysen2017}.
Additionally, we plan to implement support for the post-seismic deformation
models in ITRF2014.

The next step is to implement the possibility to create a reference frame based
on the individual session solutions. Later, we will investigate the
possibilities for solving the ionosphere and ambiguities in the next step
towards creating an independent analysis software.

\endinput
