% Why these sessions
On the 17th of February 2020 the new station NYALE13S observed its first
successful 24 hour session. The week before some data was successfully correlated,
but since it was only 88 observations spanning 3 hours the station was not included 
in the official database. There is still only a limited number of sessions available 
with observations from both NYALE13S (Ns) and NYALES20 (Ny). The few sessions that 
exist are used in this analysis and are listed in table \ref{tab:sessions}.

NYALE13S observed with a warm receiver until R1945. But the SEFD values used in the schedule was not
updated to reflect the cold receiver until R1947. At the same time the receiver
at NYALES20 started to heat up and observed with a warm receiver for the
remaining sessions. NYALE13S has been scheduled as a tag-along
station for all these sessions. NYALE13S used a DBBC3 and FlexBuff system. In
the end the DBBC3 malfunctioned and had to be sent for repairs. The DBBC3 was
later replaced with a new DBBC2, but by then the elevation encoder at NYALES20
had malfunctioned and also had to be sent for repairs.

\begin{table}
\captionsetup{labelfont={color=kvlightgreen}}
	\begin{tabularx}{\columnwidth}{X|X}
	Session & Date \\
	\hline
	R1934 & 2020 02 17 \\
	R1935 & 2020 02 24 \\
	R1936 & 2020 03 02 \\
	R1937 & 2020 03 09 \\
	\sout{R1938} & \sout{2020 03 16} \\
	R1939 & 2020 03 23 \\
	R1940 & 2020 03 30 \\
	\sout{R1941} & \sout{2020 04 06} \\
	\sout{R1942} & \sout{2020 04 14} \\
	\sout{R1943} & \sout{2020 04 20} \\
	R1944 & 2020 04 27 \\
	R1945 & 2020 05 04 \\
	R1946 & 2020 05 11 \\
	R1947 & 2020 05 18 \\
	R1948 & 2020 05 26 \\
	R1949 & 2020 06 02 \\
	\sout{R1950} & \sout{2020 06 08} \\
	R1951 & 2020 06 15 \\
	R1952 & 2020 06 22 \\
	\hline
	\end{tabularx}
\caption{\textcolor{white}{Sessions used in analysis. The struck through rows are sessions with no useful
 data from NYALE13S.}}
\label{tab:sessions}
\end{table}

\endinput
