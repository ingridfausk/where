The length of the baseline 
vector is approximately 1.5km and there is a height difference between the stations of approximately
33 meters. In addition, the NYALE13S is very close to the coast. Are these stations close
enough to experience the same troposphere? The baseline length seem to agree better with the local tie
vector if the troposphere is treated the same at both stations. But the repeatability is too poor and
the number of sessions are too few to be really sure. The uncertainty is mostly due to the perfomance of 
NYALE13S. The local tie vector is also just a premilinary 
result without known uncertainties at the moment. There are GNSS stations close to both NYALE13S and
NYALES20. Troposphere estimates from these stations should be investigated to compare the troposphere
at the two locations.  
 
Regardless, the local tie computations need to be finalized and more and better observations are needed
from the VLBI antennas. Only two sessions have good quality (R1947 and R1952) and the weighted baseline
length repeatability would be significantly better if more sessions were of this quality. The repairs at
NYALES20 are imminent and hopefully more sessions with both stations can be recorded very soon.

\endinput
