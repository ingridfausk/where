Multiple solutions were tested to see how the parameterization might affect the final results. The number of observations
available from NYALE13S is very limited and the quality is for the most part poor due to the warm receiver and other
problems with the sessions. This lowers the degrees of freedom and increases the uncertainty in the estimates. Different
setups for troposphere parameterizations and fixing the celestial reference frame have been tested.

The analysis has been done using \textbf{Where} \cite{hjelle2018}. The default solution is the same which is used for 
regular R1 and R4 processing. This means all station and source coordinates, EOPs, clocks and troposphere are estimated, 
for more information see \cite{kirkvik2017b} and \cite{kirkvik2019}. The results are summarized in table 
\ref{tab:solutions} and selected plots from the two extremes in terms of estimated parameters (solution 0 and solution 9) 
are shown in figure \ref{fig:plots}. A priori coordinates for NYALE13S used in the analysis are computed based on the local 
tie vector since the coordinates in the original database is off by almost a meter. The baseline length and repeatability is 
calculated according to \cite{hofmeister2016}.
 

\begin{table}
\captionsetup{labelfont={color=kvlightgreen}}
	\begin{tabularx}{\columnwidth}{l|l|C|C|C}
	~\#~ & ~Setup~ & WBL & WBLR & dL \\
	\hline
	~0~ & ~\makecell[tl]{Default}~ & 1539.1932 [m] & 0.0032 [m] & 0.0001 [m] \\
	~1~ & ~\makecell[tl]{Troposphere gradients fixed (Ns)}~ & 1539.1943 [m] & 0.0026 [m] & 0.0012 [m] \\
	~2~ & ~\makecell[tl]{Troposphere gradients fixed (Ns and Ny) \\ 
	Zenith wet delay fixed (Ns and Ny)}~ & 1539.1936 [m] & 0.0025 [m] & 0.0005 [m] \\
	~3~ & ~\makecell[tl]{Troposphere gradients fixed (Ns) \\ 
	Two hour zenith wet delay (Ns)}~ & 1539.1936 [m] & 0.0025 [m]  & 0.0010 [m] \\
	~4~ & ~\makecell[tl]{Troposphere gradients fixed (Ns and Ny) \\ 
	Two hour zenith wet delay (Ns and Ny)}~ & 1539.1931 [m] & 0.0021 [m] & 0.0000 [m] \\
	~5~ & ~\makecell[tl]{Radio sources fixed}~ & 1539.1939 [m] & 0.0021 [m] & 0.0008 [m] \\
	~6~ & ~\makecell[tl]{Radio sources fixed \\ 
	Troposphere gradients fixed (Ns)}~ & 1539.1945 [m] & 0.0025 [m] & 0.0014 [m] \\
	~7~ & ~\makecell[tl]{Radio sources fixed \\ 
	Troposphere gradients fixed (Ns and Ny) \\ 
	Zenith wet delay fixed (Ns and Ny)}~  & 1539.1936 [m] & 0.0027 [m] & 0.0005 [m] \\
	~8~ & ~\makecell[tl]{Radio sources fixed \\ 
	Troposphere gradients fixed (Ns) \\ 
	Two hour zenith wet delay (Ns)}~ & 1539.1943 [m] & 0.0022 [m] & 0.0012 [m] \\
	~9~ & ~\makecell[tl]{Radio sources fixed \\ 
	Troposphere gradients fixed (Ns and Ny) \\ 
	Two hour zenith wet delay (Ns and Ny)}~ & 1539.1943 [m] & 0.0022 [m] & 0.0000 [m] \\
	\hline
	\end{tabularx}
\caption{\textcolor{white}{Weighted baseline length (WBL), weighted baseline length repeatability (WBLR) and difference 
between weighted baseline length and local tie vector (dL) for the different solutions. The second column described how 
a solution differs from the default solution.
}}
\label{tab:solutions}
\end{table}

\endinput
