At the end of 2016 the European Commission has declared the Galileo Initial Services. The declaration means that the Galileo satellites and ground infrastructure are operational and ready for positioning, navigation and timing to users on the way to full operational capability in 2020.

Galileo performance monitoring plays an important role for testing and verifying the initial services and to ensure the provision of high quality satellite data to users. One of the key performance indicators is the signal-in-space range error (SISRE). SISRE represents the error budget related to the control and space segment of Global Navigation Satellite Systems and can be determined by comparing broadcast against precise ephemerides.

The SISRE analysis is implemented in the geodetic analysis software \textbf{Where}. We will describe the used methodology and we will present the first results of the Galileo orbit performance monitoring based on the SISRE analysis with \textbf{Where}.

\vspace{2cm}
\endinput
